% LaTeX-Vorlage Medizintechnik Projektarbeit
% Alexander Ruppel
% veraendert von Eva Eibenberger 
% veraendert von Stephan Seitz

% %%%%%%%%%%%%%%%%%%%%%%%%%%%%%%%%%%%%%%%%%%%%%%%%%%%%%%
% Please change the title of the project work, add your name 
% and matriculation number and set the language of your project 
% report 
% %%%%%%%%%%%%%%%%%%%%%%%%%%%%%%%%%%%%%%%%%%%%%%%%%%%%%%
\documentclass[%
	a4paper, %
	12pt, %
	english, % set to english if you want to write in English
	bibtotoc %
]{scrartcl}

% Gruppe: Nummer der Projektarbeit

% Thema der Projektarbeit
\newcommand{\titel}{Title}

% Studentenname und Matrikelnummer 
\newcommand{\erster}{Peter Pan}		% Student 1: Vorname Nachname
\newcommand{\mnreins}{1234567}		% Student 1: Matrikelnummer

% in header Spache einstellen!
\input{header}

\begin{document}

% LaTeX-Vorlage Medizintechnik Projektarbeit
% Wintersemster 2009/10
% Alexander Ruppel
% veraendert von Eva Eibenberger 

% %%%%%%%%%%%%%%%%%%%%%%%%%%%%%%%%%%%%%%%%%%%%%%%%%%%%%%
% Diese Datei muss NICHT veraendert werden
% %%%%%%%%%%%%%%%%%%%%%%%%%%%%%%%%%%%%%%%%%%%%%%%%%%%%%%

\begin{titlepage}

\begin{center}
Friedrich-Alexander-Universit\"at Erlangen-N\"urnberg\\
Department Informatik\\
Lehrstuhl f\"ur Mustererkennung\\
Prof.\ A.\ Maier\\

\vspace*{9em}

{\huge \textbf{\textsf{Medizintechnik I}}}\\[.3em]
{Projektarbeit}\\[.3em]
{Wintersemster 2016/17}\\

\vspace*{9em}

{\huge \textbf{\textsf{\titel}}}\\[.7em]
{<Datum>: z.B. 30.1.2017}
\end{center}

\vfill% {
\begin{tabbing}
	\hspace*{5cm} \= Vorname Nachname \hspace*{4em} \= Matrikelnummer \kill
	Studierender:\> \erster \> \mnreins \\
%	\ifthenelse{\equal{\student}{\erster}}{\textbf{\erster} \> \textbf{\mnreins}}{\erster \> \mnreins} \\
%	\ifthenelse{\equal{\student}{\zweiter}}{\textbf{\zweiter} \> \textbf{\mnrzwei}}{\zweiter \> \mnrzwei} \\
%	\ifthenelse{\equal{\student}{\dritter}}{\textbf{\dritter} \> \textbf{\mnrdrei}}{\dritter \> \mnrdrei} \\
%	\ifthenelse{\equal{\student}{\vierter}}{\textbf{\vierter} \> \textbf{\mnrvier}}{\vierter \> \mnrvier} \\
%	\ifthenelse{\equal{\student}{\fuenfter}}{\textbf{\fuenfter} \> \textbf{\mnrfuenf}}{\fuenfter \> \mnrfuenf} \\
\end{tabbing}
%}

\end{titlepage}


% Inhaltsverzeichnis
\tableofcontents
\newpage

% Dateien, die den Text enthalten
\section{Introduction}%
\label{sec:einleitung}
Magnetic Resonance Imaging (MRI) is \dots
% High level overview des Verfahrens 
% Invasiv?
% (1-2) sätze

\subsection{From the Physical Signal to the Image}
The image acquisition works by \dots
% phsyikalischer Hintergrund (5-7 Sätze) 
% wie kommen 
\subsection{Advantages and Disadvantages of MRI}
The advantages of MRI are \dots

The disadvantages of MRI are  \dots

\subsection{Overview}
% (3-5) sätze die einen Überblick über das Projekt geben 
In the following \dots



\section{Methods}
\subsection{k-Space}
The k-space is \dots
% Erklärung - was ist k-space und wie genau ist der Zusammenhang zu Real und Imaginärteil 

The formula to calculate the magnitude and phase is \dots
\begin{equation}
    r =  \\
\end{equation}
\begin{equation}
    \phi = 
\end{equation}

In Figure \ref{fig:real_imag} \dots
%beschreibung

In Figure \ref{fig:mag_phas} \dots
%beschreibung

If we compare both figures we can see \dots


\begin{figure}
    \centering
    %TODO add graphic and remove vspace!
    %\includegraphics{}
    \vspace{5cm}
    \caption{The Figure shows the real (left) and the imaginary part (left) of the signal.}
    \label{fig:real_imag}
\end{figure}

\begin{figure}
    \centering
    %TODO add graphic and remove vspace!
    %\includegraphics{}
    \vspace{5cm}
    \caption{The Figure shows the magnitude (left), the magnitude in logarithmic scale (middle) and the phase (right) of the k-space}
    \label{fig:mag_phas}
\end{figure}



\subsection{Reconstruction}
To reconstruct \dots
% was ist rekonstruction (1-2 sätze) 

\subsubsection{Fourier-Transformation}
The Fourier-Transformation \dots
% Was ist die fourier Transformation (1-2 Sätze) 
% Welchen Algorithmus benutzen wir zur Transformation? 
% Was versteht man unter Gleichanteil

Figure \ref{fig:Verschiebung} shows \dots
% Erkärung des Verfahrens (3-5 Sätze) 

\begin{figure}
    \centering
    %TODO add graphic and remove vspace!
    %\includegraphics{}
    \vspace{5cm}
    \caption{The Figure shows the two-dimensional fourier shift.}
    \label{fig:Verschiebung}
\end{figure}


\begin{figure}
    \centering
    %TODO add graphic and remove vspace!
    %\includegraphics{}
    \vspace{5cm}
    \caption{Reconstruction with (left) and without correctly shifted fourier components (right).}
    \label{fig:Verschiebung}
\end{figure}

\subsubsection{Reconstruction}
Figure \ref{fig:mag_re_im} shows \dots
%Was is zu sehen? 
%Wie ist es mit \ref{fig:Verschiebung} zu vergleichen? 

\begin{figure}
    \centering
    %TODO add graphic and remove vspace!
    %\includegraphics{}
    \vspace{5cm}
    \caption{The Figure shows from left to right the from the k-space reconstructed images of the phase, the real, and the imaginary part}
    \label{fig:mag_re_im}
\end{figure}

\subsubsection{Repreduction of the raw image data}
To get back to the raw image data \dots
%Was muss man genau machen? 
%Wieso geht das? 

\subsection{Filter}

After describing how to create an MR image in the previous chapters, in this chapter we will look at how the images can be modified using filters.

Filtering is \dots\\
The low frequency \dots\\
The high frequenzen \dots\\

Figure \ref{fig:filterung} shows \dots

\begin{figure}
    \centering
    %TODO add graphic and remove vspace!
    %\includegraphics{}
    \vspace{5cm}
    \caption{The Figure shows an image without high frequencies on the left and on the right without the low frequencies. The middle one is unfiltered for reference}
    \label{fig:filterung}
\end{figure}

\subsection{Sinc-Filter}
The sinc filter is a\dots %TODO
pass filter \dots 
%how does it work? 
%Kspace

Figure \ref{fig:filterungsinc} shows \dots 

\begin{figure}
    \centering
    %TODO add graphic and remove vspace!
    %\includegraphics{}
    \vspace{5cm}
    \caption{The Figure shows the magnitude image and the magnitude of the k-space after application of the sinc filters.}
    \label{fig:filterungsinc}
\end{figure}

Figure \ref{fig:filterung-kspace} shows \dots

\begin{figure}
    \centering
    %TODO add graphic and remove vspace!
    %\includegraphics{}
    \vspace{5cm}
    \caption{The Figure shows the magnitude of the k-space before (left) and after (right) application of the sinc filter on a logarithmic scale}
    \label{fig:filterung-kspace}
\end{figure}

\subsection{Box multiplikation}
Besides the Sinc filter, there are other ways to reduce the sharpness of an image, for example, by setting the edge of the frequency range to zero.
Therefore, \dots
In Figure \ref{fig:box-mulitplikation} \dots

\begin{figure}
    \centering
    %TODO add graphic and remove vspace!
    %\includegraphics{}
    \vspace{5cm}
    \caption{The figure shows that magnitude of the k-space and the reconstructed image after setting different number of frequencies to zero.}
    \label{fig:box-mulitplikation}
\end{figure}

\subsection{Comparison between sinc filter and box multiplication}
Figure \ref{fig:boxvssinc} shows the comparison between \dots

\begin{figure}
    \centering
    %TODO add graphic and remove vspace!
    %\includegraphics{}
    \vspace{5cm}
    \caption{The Figure shows the reconstructed image after application of the sinc filter (left) and zeroing out the borders (box multiplication)}
    \label{fig:boxvssinc}
\end{figure}


\subsection{Max Pooling}
Max pooling is  \dots
%Erklärung der Operation

Figure \ref{fig:maxpooling} shows \dots

\begin{figure}
    \centering
    %TODO add graphic and remove vspace!
    %\includegraphics{}
    \vspace{5cm}
    \caption{The Figure shows the comparison before (left) and after (right) application of max pooling.}
    \label{fig:maxpooling}
\end{figure}


\section{Conclusion}
Summary \dots

Current research 1 \dots
%Erklärung der Richtung und der Relevanz mit Referenz! 

Current research 2 \dots
%Erklärung der Richtung und der Relevanz mit Referenz! 



% Literaturverzeichnis
\newpage
\bibliographystyle{apalike}
\bibliography{Bib/literatur}

\end{document}
  